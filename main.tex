\documentclass{article}
\usepackage{graphicx} % Required for inserting images
\usepackage{pawlowski}
\usepackage{amsmath}
\title{Assignment 7}
\author{Joud Bamehriz}
\date{November 2024}

\begin{document}

\maketitle
\newpage
\section{Numerical Differentiation}
Taking numerical derivatives requires approximating the derivative of a function f(x) at a point x by using discrete values of f(x) at nearby points. There are many different schemes with different degrees of accuracy for numerical differentiation. 
\subsection{First Order Differencing}
\subsubsection{The Taylor Expansion}
The Taylor expansion of a function f(x) around a point x with a small increment h is:
\begin{align}
  f(x+h)=f(x)+hf'(x)+\dfrac {h^2}{2}f''(x)+...\\
  f(x-h)=f(x)-hf'(x)+\dfrac {h^2}{2}f''(x)+...
\end{align}
These formulas allow us to solve for $f'(x)$ by manipulating $f(x+h)$ and $f(x-h)$ 
\subsubsection{Forward Differencing} 
The forward differencing scheme approximates the first derivative of $f(x)$ using the function value at $x$ and $x+h$. The Taylor series expansion for $f(x+h)$ around $x$ is $$f(x+h)=f(x)+hf'(x)+\dfrac {h^2}{2}f''(x)+...$$ To approximate $f'(x)$, we rearrange this last equation solving for $f'(x)$: 
\begin{align}
  f'(x)=\dfrac {f(x+h)-f(x)}{h}-\dfrac{h}{2}f''(x)+...
\end{align}
Neglecting the higher-order terms gives the forward difference approximation: 
\begin{align}
  f'(x)\approx\dfrac {f(x+h)-f(x)}{h}
\end{align}
Since this approximation only considers first-order terms, it's called a first-order approximation and is first-order accurate. 
\subsubsection{Backward Differencing} 
Similarly to forward differencing, the backward differencing scheme approximates the first derivative of $f(x)$ using the function value at $x$ and $x-h$. The Taylor series expansion for $f(x-h)$ around $x$ is $$f(x+h)=f(x)-hf'(x)+\dfrac {h^2}{2}f''(x)+...$$ To approximate $f'(x)$, we rearrange this last equation solving for $f'(x)$: 
\begin{align}
  f'(x)=\dfrac {f(x+)-f(x-h)}{h}-\dfrac{h}{2}f''(x)+...
\end{align}
Again neglecting the higher-order terms gives the forward difference approximation: 
\begin{align}
  f'(x)\approx\dfrac {f(xh)-f(x-h)}{h}
\end{align}
\subsection{Second Order Differencing}
Schemes like this one are more accurate than the previously mentioned ones.
\subsubsection{Central Differencing}
The central differencing scheme uses values at $x+h$ \boldsymbol{and} $x-h$ to approximate $f'(x)$. We start by subtracting the Taylor expansion of $f(x+h)$ and $f(x-h)$ to get: 
\begin{align}
 f(x+h)-f(x-h)=2hf'(x)+...
\end{align}
Dividing by $2h$, we obtain: 
\begin{align}
 f'(x)\approx\dfrac{f(x+h-f(x-h}{2h}
\end{align}
\subsection{Third Order Differencing}
To derive an even more accurate scheme we use function values around $x$, including points like $f(x+h),f(x-h)$, and $f(x+2h)$. 
\\Starting with the Taylor series expansions for each and then trying to form a linear combination of these equations to isolate $f(x)$ while canceling terms up to $h^3$. A useful combination for a third-order scheme is: 
\begin{align}
 f'(x)\approx\dfrac{-3f(x)+4f(x+h)-f(x+2h)}{2h}
\end{align}
This formula 9  provides a higher accuracy than first- and second-order schemes without requiring points behind $x$, making it particularly useful for forward-based approximations.
\section{Potential of 50 Random Point Charges}
\subsection{Summary}
In this problem, we calculate the electric potential and electric field produced by a set of 50 randomly distributed point charges. The charges are generated using the qgrid function from the provided creategrid.py module, which places these charges at random locations on a grid, with each charge assigned a semi-random value. This setup simulates a complex, multi-charge system and provides a dynamic distribution of both positive and negative charges.
\subsection{Methodology}
To calculate the electric potential across the grid, we use the principle of superposition. For each point charge generated by qgrid, we compute its contribution to the electric potential at every point on the grid. The total potential is then obtained by summing these individual contributions. To determine the electric field, we take the negative gradient of the potential, which yields the field components in the x and y directions.
\subsection{Setup}
We set up the grid to cover an area ranging from -10 to 10 in the x-direction and -20 to 20 in the y-direction, ensuring that all charges are included within the grid boundaries. To avoid infinite values in the calculations where the potential is evaluated very close to a charge, we set a minimum distance threshold in the computation. This choice prevents numerical singularities and stabilizes the calculations.
\subsection{Results and Observations} 
\begin{figure}
    \includegraphics[width=0.5\textwidth]{figure.png}
    \caption{electric potential plot}
    \label{potential plot}
\end{figure}
\begin{figure}
    \includegraphics[width=0.5\textwidth]{figure2.png}
    \caption{electric field plot}
    \label{electric field plot}
\end{figure}
The resulting plots \ref{potential plot} and \ref{electric field plot} illustrate the distribution of the electric potential and electric field. The potential plot shows regions of high and low potential corresponding to the locations and magnitudes of the charges. The electric field plot, created using vector arrows, reveals the field directions and strengths, indicating regions where the field is strongest, particularly near clusters of charges. This visualization provides insight into the complex interactions among multiple charges and the resulting fields in the surrounding space.
\section{Harmonic Oscillations and Damping}
The following section explores the behavior of a damped harmonic oscillator under four different damping conditions by varying the damping coefficient C. For each case, the Euler method was used to numerically solve the oscillator’s equations of motion, generating displacement versus time plots.
\begin{description}
    \item[No Damping (C=0)] In the absence of damping, the oscillator exhibits simple harmonic motion. The displacement oscillates sinusoidally with constant amplitude, as no energy is dissipated. This case represents an ideal scenario where the system oscillates indefinitely due to the lack of resistive forces, check figure \ref{undamped}.
    \item[Criticial Damping (c=$4\sqrt{km}$)] For the critically damped case, the damping coefficient is adjusted to just prevent oscillations. Here, the system returns to equilibrium as quickly as possible without overshooting. This behavior is desirable in applications where minimizing oscillations is important, as the system stabilizes in the shortest time, check figure \ref{critically}.
    \item[Underdamped (c$<4\sqrt{km}$)] When the damping is set below the critical threshold, the system oscillates but gradually decreases in amplitude due to energy loss over time. This is a common scenario in real-world oscillators, where some damping is present, but the system still oscillates as it returns to equilibrium, check figure \ref{under}. 
    \item[Overdamped (c$>4\sqrt{km}$] In the overdamped case, the system is highly resistive, causing it to return to equilibrium slowly without oscillating. The displacement decays exponentially towards zero. This scenario is observed in systems where strong damping is applied, such as heavy shock absorbers, check figure \ref{over}
\end{description}
Each figure below demonstrates the displacement over time for the respective damping condition:
\begin{figure}[h!] 
    \includegraphics[width=0.45\textwidth]{harmonic_0.png} \caption{No Damping (C = 0)}
    \label{undamped}
\end{figure}
 \begin{figure}[h!]  
    \includegraphics[width=0.45\textwidth]{harmonic_critical.png} \caption{Critically Damped (C = 4$\sqrt{km}$)} 
    \label{critically}
\end{figure} 
\begin{figure}[h!] 
    \includegraphics[width=0.45\textwidth]{harmonic_under.png} \caption{Underdamped (C < 4$\sqrt{km}$)} 
    \label{under}
\end{figure} 
\begin{figure}[h!] 
    \includegraphics[width=0.45\textwidth]{harmonic_over.png} \caption{Overdamped (C > 4$\sqrt{km}$)} 
    \label{over}
\end{figure} 
\section{Cycling Without Drag}
\begin{figure}[h!] 
    \includegraphics[width=0.45\textwidth]{cycling.png} 
    \label{cycling}
\end{figure} 
The resulting plot of velocity versus time figure \ref{cycling} shows that the cyclist’s velocity increases initially due to the constant power output. However, as the velocity increases, the rate of acceleration decreases because $dv/dt$ is inversely proportional to $v$. This reflects the fact that, with constant power output, less force is available to accelerate the cyclist at higher velocities. Thus, the cyclist’s velocity asymptotically approaches a terminal speed where all power output is used to maintain the velocity rather than increase it. 
\end{document}
